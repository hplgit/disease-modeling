%%
%% Automatically generated file from Doconce source
%% (https://github.com/hplgit/doconce/)
%%
% #ifdef PTEX2TEX_EXPLANATION
%%
%% The file follows the ptex2tex extended LaTeX format, see
%% ptex2tex: http://code.google.com/p/ptex2tex/
%%
%% Run
%%      ptex2tex myfile
%% or
%%      doconce ptex2tex myfile
%%
%% to turn myfile.p.tex into an ordinary LaTeX file myfile.tex.
%% (The ptex2tex program: http://code.google.com/p/ptex2tex)
%% Many preprocess options can be added to ptex2tex or doconce ptex2tex
%%
%%      ptex2tex -DMINTED myfile
%%      doconce ptex2tex myfile envir=minted
%%
%% ptex2tex will typeset code environments according to a global or local
%% .ptex2tex.cfg configure file. doconce ptex2tex will typeset code
%% according to options on the command line (just type doconce ptex2tex to
%% see examples). If doconce ptex2tex has envir=minted, it enables the
%% minted style without needing -DMINTED.
% #endif

% #define PREAMBLE

% #ifdef PREAMBLE
%-------------------- begin preamble ----------------------

\documentclass[%
twoside,                 % oneside: electronic viewing, twoside: printing
final,                   % or draft (marks overfull hboxes, figures with paths)
10pt]{article}

\listfiles               % print all files needed to compile this document

\usepackage{relsize,epsfig,makeidx,color,setspace,amsmath,amsfonts}
\usepackage[table]{xcolor}
\usepackage{bm,microtype}

\usepackage{ptex2tex}

% #ifdef MINTED
\usepackage{minted}
\usemintedstyle{default}
% #endif

\usepackage[T1]{fontenc}
%\usepackage[latin1]{inputenc}
\usepackage[utf8]{inputenc}

\usepackage{lmodern}         % Latin Modern fonts derived from Computer Modern

% Hyperlinks in PDF:
\definecolor{linkcolor}{rgb}{0,0,0.4}
\usepackage[%
    colorlinks=true,
    linkcolor=linkcolor,
    urlcolor=linkcolor,
    citecolor=black,
    filecolor=black,
    %filecolor=blue,
    pdfmenubar=true,
    pdftoolbar=true,
    bookmarksdepth=3   % Uncomment (and tweak) for PDF bookmarks with more levels than the TOC
            ]{hyperref}
%\hyperbaseurl{}   % hyperlinks are relative to this root

\setcounter{tocdepth}{2}  % number chapter, section, subsection

% Tricks for having figures close to where they are defined:
% 1. define less restrictive rules for where to put figures
\setcounter{topnumber}{2}
\setcounter{bottomnumber}{2}
\setcounter{totalnumber}{4}
\renewcommand{\topfraction}{0.85}
\renewcommand{\bottomfraction}{0.85}
\renewcommand{\textfraction}{0.15}
\renewcommand{\floatpagefraction}{0.7}
% 2. ensure all figures are flushed before next section
\usepackage[section]{placeins}
% 3. enable begin{figure}[H] (often leads to ugly pagebreaks)
%\usepackage{float}\restylefloat{figure}

\usepackage[framemethod=TikZ]{mdframed}

% --- begin definitions of admonition environments ---

% Admonition style "mdfbox" is an oval colored box based on mdframed
% "notice" admon
\colorlet{mdfbox_notice_background}{gray!5}
\newmdenv[
  skipabove=15pt,
  skipbelow=15pt,
  outerlinewidth=0,
  backgroundcolor=mdfbox_notice_background,
  linecolor=black,
  linewidth=2pt,       % frame thickness
  frametitlebackgroundcolor=mdfbox_notice_background,
  frametitlerule=true,
  frametitlefont=\normalfont\bfseries,
  shadow=false,        % frame shadow?
  shadowsize=11pt,
  leftmargin=0,
  rightmargin=0,
  roundcorner=5,
  needspace=0pt,
]{notice_mdfboxmdframed}

\newenvironment{notice_mdfboxadmon}[1][]{
\begin{notice_mdfboxmdframed}[frametitle=#1]
}
{
\end{notice_mdfboxmdframed}
}

% Admonition style "mdfbox" is an oval colored box based on mdframed
% "summary" admon
\colorlet{mdfbox_summary_background}{gray!5}
\newmdenv[
  skipabove=15pt,
  skipbelow=15pt,
  outerlinewidth=0,
  backgroundcolor=mdfbox_summary_background,
  linecolor=black,
  linewidth=2pt,       % frame thickness
  frametitlebackgroundcolor=mdfbox_summary_background,
  frametitlerule=true,
  frametitlefont=\normalfont\bfseries,
  shadow=false,        % frame shadow?
  shadowsize=11pt,
  leftmargin=0,
  rightmargin=0,
  roundcorner=5,
  needspace=0pt,
]{summary_mdfboxmdframed}

\newenvironment{summary_mdfboxadmon}[1][]{
\begin{summary_mdfboxmdframed}[frametitle=#1]
}
{
\end{summary_mdfboxmdframed}
}

% Admonition style "mdfbox" is an oval colored box based on mdframed
% "warning" admon
\colorlet{mdfbox_warning_background}{gray!5}
\newmdenv[
  skipabove=15pt,
  skipbelow=15pt,
  outerlinewidth=0,
  backgroundcolor=mdfbox_warning_background,
  linecolor=black,
  linewidth=2pt,       % frame thickness
  frametitlebackgroundcolor=mdfbox_warning_background,
  frametitlerule=true,
  frametitlefont=\normalfont\bfseries,
  shadow=false,        % frame shadow?
  shadowsize=11pt,
  leftmargin=0,
  rightmargin=0,
  roundcorner=5,
  needspace=0pt,
]{warning_mdfboxmdframed}

\newenvironment{warning_mdfboxadmon}[1][]{
\begin{warning_mdfboxmdframed}[frametitle=#1]
}
{
\end{warning_mdfboxmdframed}
}

% Admonition style "mdfbox" is an oval colored box based on mdframed
% "question" admon
\colorlet{mdfbox_question_background}{gray!5}
\newmdenv[
  skipabove=15pt,
  skipbelow=15pt,
  outerlinewidth=0,
  backgroundcolor=mdfbox_question_background,
  linecolor=black,
  linewidth=2pt,       % frame thickness
  frametitlebackgroundcolor=mdfbox_question_background,
  frametitlerule=true,
  frametitlefont=\normalfont\bfseries,
  shadow=false,        % frame shadow?
  shadowsize=11pt,
  leftmargin=0,
  rightmargin=0,
  roundcorner=5,
  needspace=0pt,
]{question_mdfboxmdframed}

\newenvironment{question_mdfboxadmon}[1][]{
\begin{question_mdfboxmdframed}[frametitle=#1]
}
{
\end{question_mdfboxmdframed}
}

% Admonition style "mdfbox" is an oval colored box based on mdframed
% "block" admon
\colorlet{mdfbox_block_background}{gray!5}
\newmdenv[
  skipabove=15pt,
  skipbelow=15pt,
  outerlinewidth=0,
  backgroundcolor=mdfbox_block_background,
  linecolor=black,
  linewidth=2pt,       % frame thickness
  frametitlebackgroundcolor=mdfbox_block_background,
  frametitlerule=true,
  frametitlefont=\normalfont\bfseries,
  shadow=false,        % frame shadow?
  shadowsize=11pt,
  leftmargin=0,
  rightmargin=0,
  roundcorner=5,
  needspace=0pt,
]{block_mdfboxmdframed}

\newenvironment{block_mdfboxadmon}[1][]{
\begin{block_mdfboxmdframed}[frametitle=#1]
}
{
\end{block_mdfboxmdframed}
}

% --- end of definitions of admonition environments ---

% prevent orhpans and widows
\clubpenalty = 10000
\widowpenalty = 10000

% --- end of standard preamble for documents ---


% insert custom LaTeX commands...

\raggedbottom
\makeindex

%-------------------- end preamble ----------------------

\begin{document}

% #endif


% ------------------- main content ----------------------



% ----------------- title -------------------------

\title{Mathematical Modeling of the Spreading of Diseases}

% ----------------- author(s) -------------------------

\author{Hans Petter Langtangen\inst{1,2}}
\institute{Center for Biomedical Computing, Simula Research Laboratory\inst{1}
\and
University of Oslo, Dept.~of Informatics\inst{2}}
% ----------------- end author(s) -------------------------

\date{Aug 14, 2014
% <optional titlepage figure>
}

\begin{center}  % inline figure
  \centerline{\includegraphics[width=0.9\linewidth]{fig/disease1.jpg}}
\end{center}


% !split
\subsection{A very complex phenomenon is modeled by simple math....}


\begin{block_mdfboxadmon}[Assumptions:]
\begin{itemize}
 \item We have a perfectly mixed population in a confined area

 \item We do not consider spatial movements, just how the disease
   evolves in time

 \item We do not consider individuals, just a grand mix of them\\
   (cf.~statistical mechanics vs thermodynamics)
\end{itemize}

\noindent
\end{block_mdfboxadmon}



% !bpop
We consider very simple models, but these can be extended to full
models that are used world-wide by health authorities. Typical
diseases: flu, measles, swine flu, HIV, ...
% !epop

% !split
\subsection{We keep track of 3 categories}


\begin{block_mdfboxadmon}[Categories (SIR model):]
\begin{itemize}
 \item \textbf{S}: susceptibles - who can get the disease

 \item \textbf{I}: infected - who have developed the disease and infect susceptibles

 \item \textbf{R}: recovered - who have recovered and become immune
\end{itemize}

\noindent
\end{block_mdfboxadmon}




\begin{block_mdfboxadmon}[Mathematical quantities:]
$S(t)$, $I(t)$, $R(t)$ (no of people).
\end{block_mdfboxadmon}




\begin{block_mdfboxadmon}[Goal:]
Find and solve equations for $S(t)$, $I(t)$, $R(t)$.
\end{block_mdfboxadmon}



% !split
\subsection{$\Delta t\,\beta SI$ people move from S to I in a time inverval $\Delta t$}


\begin{block_mdfboxadmon}[S-I interaction:]
\begin{itemize}
  \item In a mix of S and I people, there are $SI$ possible pairs

  \item A certain fraction $\Delta t\,\beta$ of $SI$ meet in a (small)
    time interval $\Delta t$, with the result that the infected
    ``successfully'' infects the susceptible

  \item The loss $\Delta t\,\beta SI$ in the S catogory is a corresponding
    gain in the I category
\end{itemize}

\noindent
\end{block_mdfboxadmon}




\begin{block_mdfboxadmon}[Remark.]
\vspace{0.5mm}\par\noindent
{\footnotesize It is reasonable that the fraction depends on $\Delta t$, and
$\beta$ is some unknown parameter we must measure, supposed to not
depend on $\Delta t$, but maybe time $t$.
\par}
\end{block_mdfboxadmon}



% !split
\subsection{For practical calculations, we must express the S-I interaction with symbols}

Loss in $S(t)$:

\[ S(t+\Delta t) = S(t) - \Delta t\,\beta S(t)I(t)\]

Gain in $I(t)$:

\[ I(t+\Delta t) = I(t) + \Delta t\,\beta S(t)I(t)\]

% !split
\subsection{Modeling the interaction between R and I}


\begin{block_mdfboxadmon}[R-I interaction:]
\begin{itemize}
 \item After some days, the infected has recovered and moves to the R category

 \item A simple model: in a small time $\Delta t$ (say 1 day),
   a fraction $\Delta t\,\nu$ of the infected are removed
   ($\nu$ must be measured)
\end{itemize}

\noindent
\end{block_mdfboxadmon}



We must subtract this fraction in the balance equation for $I$:

\[ I(t+\Delta t) = I(t) + \Delta t\,\beta S(t)I(t) -\Delta t\,\nu I(t) \]

The loss $\Delta t\,\nu I$ is a gain in $R$:

\[ R(t+\Delta t) = R(t) + \Delta t\,\nu R(t)\]


% !split
\subsection{We have three equations for $S$, $I$, and $R$}

\begin{align}
S(t+\Delta t) &= S(t) - \Delta t\,\beta S(t)I(t)
\label{SIR1:S}\\
I(t+\Delta t) &= I(t) + \Delta t\,\beta S(t)I(t) -\Delta t\nu I(t)
\label{SIR1:I}\\
R(t+\Delta t) &= R(t) + \Delta t\,\nu R(t)
\label{SIR1:R}
\end{align}


\begin{center}  % inline figure
  \centerline{\includegraphics[width=0.7\linewidth]{fig/categories_SIR.png}}
\end{center}


Before we can compute with these, we must

\begin{itemize}
 \item know $\beta$ and $\nu$

 \item know $S(0)$ (many), $I(0)$ (few), $R(0)$ (0?)

 \item choose $\Delta t$
\end{itemize}

\noindent
% !split
\subsection{The computation involves just simple arithmetics}

\begin{itemize}
 \item Set $\Delta t=6$ minutes

 \item Set $\beta =0.0013$, $\nu =0.8333$

 \item Set $S(0)=50$, $I(1)$, $R(0)=0$
\end{itemize}

\noindent
\begin{align*}
S(\Delta t) &= S(0) - \Delta t\,\beta S(0)I(0)\approx 49.99\\
I(\Delta t) &= I(0) + \Delta t\,\beta S(0)I(0) -\Delta t\,\nu I(0)\approx 1.002\\
R(\Delta t) &= R(0) + \Delta t\,\nu R(0)\approx 0.0008333
\end{align*}

% !bpop
\begin{itemize}
 \item In reality, $S$, $I$, $R$ are integers and events are discrete (meet, get sick)

 \item In the model, we work with real numbers and continuous events

 \item Reasonable approximation in a not too small population
\end{itemize}

\noindent
% !epop

% !split
\subsection{And we can continue...}

\begin{align*}
S(2\Delta t) &= S(\Delta t) - \Delta t\,\beta S(\Delta t)I(\Delta t)\approx 49.87\\
I(2\Delta t) &= I(\Delta t) + \Delta t\,\beta S(\Delta t)I(\Delta t) -\Delta t\,\nu I(\Delta t)\approx 1.011\\
R(2\Delta t) &= R(\Delta t) + \Delta t\,\nu R(\Delta t)\approx 0.00167
\end{align*}

Repeat...

\begin{align*}
S(3\Delta t) &= S(2\Delta t) - \Delta t\,\beta S(2\Delta t)I(2\Delta t)\approx 49.98\\
I(3\Delta t) &= I(2\Delta t) + \Delta t\,\beta S(2\Delta t)I(2\Delta t) -\Delta t\,\nu I(2\Delta t)\approx 1.017\\
R(3\Delta t) &= R(2\Delta t) + \Delta t\,\nu R(2\Delta t)\approx 0.0025
\end{align*}

% !bpop
But this is getting boring! Let's ask a computer to do the work!
% !epop



% !split
\subsection{First, some handy notation}

$S^n = S(n\Delta t)$,
$I^n = I(n\Delta t)$, $R^n = R(n\Delta t)$.

The equations can now be written as

\begin{align}
S^{n+1} &= S^n - \Delta t\,\beta S^nI^n
\label{SIR1:Sc}\\
I^{n+1} &= I^n + \Delta t\,\beta S^nI^n -\Delta t\,\nu I^n
\label{SIR1:Ic}\\
R^{n+1} &= R^n + \Delta t\,\nu R^n
\label{SIR1:Rc}
\end{align}

% !split
\subsection{We variables, arrays, and a loop we can program}

Suppose we want to compute until $t=N\Delta t$, i.e., for $n=0,1,\ldots,N-1$.
We can store $S^0, S^1, S^2, \ldots, S^N$ in an array (or list).

Python (Matlab):

\bpycod
t = linspace(0, N*dt, N+1)  # all time points
S = zeros(N+1)
I = zeros(N+1)
R = zeros(N+1)

for n in range(N):
    S[n+1] = S[n] - dt*beta*S[n]*I[n]
    I[n+1] = I[n] + dt*beta*S[n]*I[n] - dt*nu*I[n]
    R[n+1] = R[n] + dt*nu*I[n]
\epycod

% !split
\subsection{Here is the complete program}

Let time be measured in hours.

\bpycod
beta = 0.0013
nu =0.8333
dt = 0.1             # 6 min
D = 30               # simulate for D days
N = int(D*24/dt)     # corresponding no of hours

from numpy import zeros, linspace
t = linspace(0, N*dt, N+1)
S = zeros(N+1)
I = zeros(N+1)
R = zeros(N+1)

for n in range(N):
    S[n+1] = S[n] - dt*beta*S[n]*I[n]
    I[n+1] = I[n] + dt*beta*S[n]*I[n] - dt*nu*I[n]
    R[n+1] = R[n] + dt*nu*I[n]

# Plot the graphs
from matplotlib.pyplot import *
plot(t, S, 'k-', t, I, 'b-', t, R, 'r-')
legend(['S', 'I', 'R'], loc='lower right')
xlabel('hours')
show()
\epycod

% !split
\subsection{We have predicted a disease!}


\begin{center}  % inline figure
  \centerline{\includegraphics[width=0.9\linewidth]{fig/SIR1.pdf}}
\end{center}



% !split
\subsection{How much math and programming did we use?}

\begin{itemize}
 \item Plain arithmetics

 \item The concept of a graph (i.e., discrete function in time)

 \item Units

 \item Variable

 \item Array

 \item Loop

 \item Plotting
\end{itemize}

\noindent
% !split
\subsection{Detour: The standard mathematical approach}

We had from intuition established

\begin{align*}
S(t+\Delta t) &= S(t) - \Delta t\,\beta S(t)I(t)\\
I(t+\Delta t) &= I(t) + \Delta t\,\beta S(t)I(t) -\Delta t\,\nu I(t)\\
R(t+\Delta t) &= R(t) + \Delta t\,\nu R(t)
\end{align*}

The mathematician will now make a \emph{differential equations}. First,
divide by $\Delta t$ and move $S$, $I$, and $R$ to the left-hand side:

\begin{align*}
\frac{S(t+\Delta t) - S(t)}{\Delta t} &= - \beta S(t)I(t)\\
\frac{I(t+\Delta t) - I(t)}{\Delta t} &= \beta t S(t)I(t) -\nu I(t)\\
\frac{R(t+\Delta t) - R(t)}{\Delta t} &= \nu R(t)
\end{align*}

% !split
\subsection{A derivative arises as $\Delta t\rightarrow 0$}

In any calculus book, the derivative of $S$ at $t$ is defined as

\[ S'(t) = \lim_{t\rightarrow 0}\frac{S(t+\Delta t) - S(t)}{\Delta t}\]

If we let $\Delta t\rightarrow 0$, we get derivatives on the left-hand side:

\begin{align*}
S'(t) &= - \beta S(t)I(t)\\
I'(t) &= \beta t S(t)I(t) -\nu I(t)\\
R'(t) &= \nu R(t)
\end{align*}

This is a 3x3 system of differential equations for the functions
$S(t)$, $I(t)$, $R(t)$. For a unique solution, we need
$S(0)$, $I(0)$, $R(0)$.

% !split
\subsection{Bad news: we cannot solve these equations!}


\begin{block_mdfboxadmon}[Time to ask a numerical methods expert:]
Replace the derivative with a \emph{finite difference}, e.g.,

\[ S'(t) \approx \frac{S(t+\Delta t) - S(t)}{\Delta t}\]
which is accurate for small $\Delta t$.
\end{block_mdfboxadmon}



This brings us back to the first model, which we can solve
on a computer!

% !split
\subsection{Parameter estimation is needed for predictive modeling}

\begin{itemize}
 \item Any small $\Delta t$ will do

 \item One can reason about $\nu$ and say that $1/\nu$ is the mean
   recovery time for the disease (e.g., 1 week for a flu)

 \item $\beta$ must in some way be measured, but we don't know what it means...
\end{itemize}

\noindent

\begin{block_mdfboxadmon}[So what if we don't know $\beta$?]
\begin{itemize}
 \item Can still learn about the \emph{dynamics} of diseases

 \item Can find the sensitivity to and influence of $\beta$

 \item Can apply \emph{parameter estimation} procedures to fit $\beta$ to data
\end{itemize}

\noindent
\end{block_mdfboxadmon}



% !split
\subsection{Let us extend the model: no life-long immunity}


\begin{block_mdfboxadmon}[Assumption.]
After some time, people in the R category lose the immunity.
In a small time $\Delta t$ this gives a leakage $\Delta t\,\gamma R$
to the S category. ($1/\gamma$ is the mean time for immunity.)
\end{block_mdfboxadmon}




\begin{center}  % inline figure
  \centerline{\includegraphics[width=0.7\linewidth]{fig/categories_SIR_feedback.png}}
\end{center}


\begin{align}
S^{n+1} &= S^n - \Delta t\,\beta S^nI^n + \color{red}{\Delta t\,\gamma R^n}
\label{SIR2:S}\\
I^{n+1} &= I^n + \Delta t\,\beta S^nI^n -\Delta t\,\nu I^n
\label{SIR2:I}\\
R^{n+1} &= R^n + \Delta t\,\nu R^n - \color{red}{\Delta t\,\gamma R^n}
\label{SIR2:R}
\end{align}

No complications in the computational model!

% !split
\subsection{The effect of loss of immunity}

$1/\gamma = 50$ days. $\beta$ reduced by 2 and 4:


\begin{center}  % inline figure
  \centerline{\includegraphics[width=0.9\linewidth]{fig/SIR2.pdf}}
\end{center}


% !split
\subsection{What is the effect of vaccination?}


\begin{block_mdfboxadmon}[Assumptions.]
A fraction $p$ of the S category, per time unit, is vaccinated with
success. Then in time $\Delta t$, $p\Delta t S$ will move to a
vaccinated category, V. This does not affect the I and R categories.
\end{block_mdfboxadmon}




\begin{center}  % inline figure
  \centerline{\includegraphics[width=0.4\linewidth]{fig/categories_SIRV.png}}
\end{center}


\begin{align}
S^{n+1} &= S^n - \Delta t\,\beta S^nI^n + \Delta t\,\gamma R^n - \color{red}{p\Delta t S^n}
\label{SIR3:S}\\
V^{n+1} &= V^n + \color{red}{p\Delta t S^n}
\label{SIR3:V}\\
I^{n+1} &= I^n + \Delta t\,\beta S^nI^n -\Delta t\,\nu I^n
\label{SIR3:I}\\
R^{n+1} &= R^n + \Delta t\,\nu R^n - \Delta t\,\gamma R^n
\label{SIR3:R}
\end{align}

Implementation: Just add array for $V^n$ and add equation.

% !split
\subsection{Many possibilities for adjusting the model...}

The effect of vaccination decreases, so we may move people back to
the S category (term proportional to $\Delta t V$).


\begin{center}  % inline figure
  \centerline{\includegraphics[width=0.7\linewidth]{fig/categories_SIRV_feedback.png}}
\end{center}



% !split
\subsection{Effect of adding vaccination}


\begin{center}  % inline figure
  \centerline{\includegraphics[width=0.8\linewidth]{fig/SIRV1.pdf}}
\end{center}


($p=0.005$)


% !split
\subsection{What is the effect of an intensive vaccination campaign?}

10 times more intense vaccination for 10 days, 6 days after outbreak:

\begin{equation*} p(t) = \left\lbrace\begin{array}{ll}
0.05,& 6\leq t\leq 15,\\
0,& \hbox{otherwise} \end{array}\right.\end{equation*}

Implementation: Let $p^n$ be an array as $V^n$. Set $p^n=0.05$ for
$n=6\cdot 24/0.1,\ldots, 15\cdot 24/0.1$ ($\mbox{days}\cdot 24 \mbox{h per
day}\Delta t$).

% !split
\subsection{Effect of vaccination campaign}


\begin{center}  % inline figure
  \centerline{\includegraphics[width=0.8\linewidth]{fig/SIRV2.pdf}}
\end{center}


Could now let the computer run a lot of cases and find the optimal
vaccination period.

% !split
\subsection{We can experiment with other campaigns}

% !bslidecell 00 0.3

\begin{center}  % inline figure
  \centerline{\includegraphics[width=0.9\linewidth]{fig/disease2.jpg}}
\end{center}

% !eslidecell

% !bslidecell 01 0.7
Masks lower $\beta$:

\begin{equation*} \beta(t) = \left\lbrace\begin{array}{ll}
\beta_1,& 0\leq t < 5,\\
\beta_2 < \beta_1,& t \geq 5\end{array}\right.
\end{equation*}

Very easy to implement. (Used to be complicated in differential
equation models...)
% !eslidecell

% !split
\subsection{And now for something similar: zombification}



\begin{center}  % inline figure
  \centerline{\includegraphics[width=0.9\linewidth]{fig/zombie1.jpg}}
\end{center}


\textbf{Zombification}: The disease that turns you into a zombie.

% !split
\subsection{Zombie modeling is almost the same as SIR modeling}


\begin{block_mdfboxadmon}[Categories.]
\begin{enumerate}
 \item S: susceptible humans who can become zombies

 \item I: infected humans, being bitten by zombies

 \item Z: zombies

 \item R: removed individuals, either conquered zombies or dead humans
\end{enumerate}

\noindent
\end{block_mdfboxadmon}



Mathematical quantities: $S(t)$, $I(t)$, $Z(t)$, $R(t)$

Zombie movie: \emph{The Night of the Living Dead}, Geoerge A. Romero, 1968

% !split
\subsection{Dynamics of the zombie SIZR model}


\begin{center}  % inline figure
  \centerline{\includegraphics[width=0.4\linewidth]{fig/categories_SIZR.png}}
\end{center}


% !bpop
\begin{enumerate}
 \item Susceptibles are infected by zombies: $-\Delta t\beta SZ$ in time $\Delta t$ (cf.~the $\Delta t\,\beta SI$ term in the SIR model).

 \item Susceptibles die naturally or get killed and then enter the removed category. The no of deaths in time $\Delta t$ is $\Delta t\delta_S S$.

 \item We also allow new humans to enter the area with zombies (necessary in a war on zombies): $\Delta t\Sigma$ during a time $\Delta t$.

 \item Some infected turn into zombies (Z): $\Delta t\rho I$, while others die (R): $\delta_I\Delta t I$.

 \item Nobody from R can turn into Z (important - otherwise zombies win).

 \item Killed zombies go to R: $\Delta t\alpha SZ$.
\end{enumerate}

\noindent
% !epop

% !split
\subsection{The four equations in the SIZR model for zombification}

\begin{align*}
S^{n+1} &= S^n + \Delta t\,\Sigma - \Delta t\,\beta S^nZ - \Delta t\,\delta_S S^n\\
I^{n+1} &= I^n + \Delta t\,\beta S^nZ^n - \Delta t\,\rho I^n - \Delta t\,\delta_I I^n\\
Z^{n+1} &= Z^n + \Delta t\,\rho I^n - \Delta t\,\alpha S^nZ^n\\
R^{n+1} &= R^n + \Delta t\,\delta_S S^n  + \Delta t\,\delta_I I^n +
\Delta t\,\alpha S^nZ^n
\end{align*}


\begin{block_mdfboxadmon}[Interpretation of parameters:]
\vspace{0.5mm}\par\noindent
{\footnotesize 
\begin{itemize}
  \item $\Sigma$: no of new humans brought into the zombified area per unit time.

  \item $\beta$: the probability that a theoretically possible human-zombie pair actually meets physically, during a unit time interval, with the result that the human is infected.

  \item $\delta_S$: the probability that a susceptible human is killed or dies, in a unit time interval.

  \item $\delta_I$: the probability that an infected human is killed or dies, in a unit time interval.

  \item $\rho$: the probability that an infected human is turned into a zombie, during a unit time interval.

  \item $\alpha$: the probability that, during a unit time interval, a theoretically possible human-zombie pair fights and the human kills the zombie.
\end{itemize}

\noindent
\par}
\end{block_mdfboxadmon}



% !split
\subsection{Simulate a zombie movie!}

% !bslidecell 00 0.5

\begin{block_mdfboxadmon}[Three fundamental phases.]
\begin{enumerate}
\item The initial phase (4 h)

\item The hysteric phase (24 h)

\item The counter attack phase (5 h)
\end{enumerate}

\noindent
\end{block_mdfboxadmon}


% !eslidecell

% !bslidecell 01 0.5

\begin{center}  % inline figure
  \centerline{\includegraphics[width=0.9\linewidth]{fig/TNotLD.pdf}}
\end{center}

% !eslidecell

% !bpop
How do we do this? As $p$ in the vaccination campaign - the parameters
take on different constant values in different time intervals.
% !epop

% !bpop
H. P. Langtangen and K.-A. Mardal and P. Røtnes:
Escaping the Zombie Threat by Mathematics, in
A. Whelan et al.: \emph{Zombies in the Academy - Living Death in Higher Education},
University of Chicago Press, 2013
% !epop

% !split
\subsection{Effective war on zombies}

Introduce attacks on zombies at selected times $T_0, T_1, \ldots, T_m$.

Model: Replace $\alpha$ by

\[ \alpha_0 + \omega (t),\]
where $\alpha_0$ is constant and $\omega(t)$ is a series of
Gaussian functions (peaks) in time:

\[ \omega(t) = a\sum_{i=0}^m \exp{\left(-\frac{1}{2}\left({t - T_i\over\sigma}\right)\right)}
\]

Must experiment with values of $a$ (strength), $\sigma$ (duration is $6\sigma$),
point of attacks ($T_i$)

% !split
\subsection{Summary}

\begin{itemize}
 \item A complex spreading is diseases can be modeled by intuitive, simple
   accounting of movement between categories

 \item Such models are knowns as \emph{compartment models}

 \item Result: difference equations that are easy to simulate on a computer

 \item (Can let $\Delta t\rightarrow 0$ and get differential equations)

 \item Easy to add new effects (vaccination, campaigns, zombification)
\end{itemize}

\noindent

% ------------------- end of main content ---------------


% #ifdef PREAMBLE
\printindex

\end{document}
% #endif

